\documentclass[13pt]{beamer}

\usepackage{pgfpages}
\usepackage{minted}
\usepackage[T1]{fontenc}
\usepackage{lmodern}
\usepackage[english]{babel}
\usepackage[utf8x]{inputenc}
\usepackage{graphicx}
\usetheme{default} % or try Darmstadt, Madrid, Warsaw, ...
\usecolortheme{default} % or try albatross, beaver, crane, ...
\usefonttheme{default} % or try serif, structurebold, ...
\setbeamertemplate{navigation symbols}{}
\setbeamertemplate{caption}[numbered]


% For presenter notes
%\setbeameroption{show notes on second screen}


\addtobeamertemplate{navigation symbols}{}{%
  \usebeamerfont{footline}%
  \usebeamercolor[fg]{footline}%
  \hspace{1em}%
  \insertframenumber/\inserttotalframenumber
  }

\newcommand\Wider[2][3em]{%
\makebox[\linewidth][c]{%
  \begin{minipage}{\dimexpr\textwidth+#1\relax}
  \raggedright#2
  \end{minipage}%
  }%
}
  
\title{Language agnostic pre-processor}
\author{Daniel Reigada - 82064}
\institute{IST - Advanced Programming}
\date{May 14, 2018}

\begin{document}

\begin{frame}
  \titlepage
  \note{
    I will start by presenting how I implemented my solution and then I will show the extensions I've made
  }
\end{frame}

\section{Introduction}

\begin{frame}[fragile]{Token definition and storage}
  \note{
    The tokens are stored in a list of pairs,
    that contain the token string and the function
  }
  \begin{minted}[fontsize=\footnotesize]{racket}
    (define *token-list* '())

    (define (add-active-token token function)
        (set! *token-list* (cons (cons token function) *token-list*)))

    (define-syntax-rule (def-active-token token parameters body)
        (add-active-token token (lambda parameters body)))

    (define (process-string str) (proccess-aux str ""))

  \end{minted}
\end{frame}

\begin{frame}[fragile]{Process function}
  \note{
    The processing function is calls it self recursivly until it
    reaches the end of the string, advancing one character at a time
    until it finds a token.
    When a token is found the respective function is called and its
    output is used to continue the recursive process.
  }
  \begin{minted}[fontsize=\footnotesize]{racket}
    (define (proccess-aux str acc)
      (cond
        [(equal? str "") acc]
        [(string-ends-with-non-whitespace? acc)
          (proccess-aux (substring str 1)
            (string-append acc (substring str 0 1)))]
        [else
          (match (find-token str)
            ['()
              (proccess-aux (substring str 1)
                (string-append acc (substring str 0 1)))] 
            [(cons token token-function)
              (proccess-aux
                (token-function (substring str (string-length token))
                acc)])]))
 \end{minted}
\end{frame}

\begin{frame}[fragile]{Finding tokens}
  \note{
    Another recursive function that checks if the string starts with
    the first token in the list, if not call itself with the tail of the list
    until the token list is empty
  }
  \begin{minted}[fontsize=\footnotesize]{racket}
    (define/match (find-token-aux str token-list)
      [(_ '()) '()]
      [(str (cons token-pair tail-list))
        (if (string-prefix? str (car token-pair))
          token-pair
          (find-token-aux str tail-list)
        )])
 \end{minted}
\end{frame}


\begin{frame}[fragile]{Local Type Inference}
  \note{
    
  }
  \begin{minted}[fontsize=\footnotesize]{racket}
    (def-active-token "var " (str) 
      (match str
        [(regexp #px".*?=.*?new (.*?)\\(.*\\).*" (list _ type))
          (string-append type " " str)]
        [str (error
          (string-append "Invalid `var` syntax in: var " str))]))
 \end{minted}
\end{frame}


\begin{frame}[fragile]{Local Type Inference}
  \note{
    
  }
  \begin{minted}[fontsize=\footnotesize]{racket}
    (define (match-string-start-end str)
      (car (regexp-match-positions
        #px"\"(?:(?=(\\\\?))\\1.)*?\""
        str)))

    (def-active-token "#" (str)
      (match-let* ([(cons start end) (match-string-start-end str)]
                    [first-string (substring str start end)]
                    [tail-string (substring str end)])
          (string-append 
            (regexp-replace* #rx"\\#\\{(.*?)\\}"
              first-string
              "\" + (\\1) + \"")
            tail-string)))

 \end{minted}
\end{frame}

\begin{frame}[fragile]{Type Aliases}
  \note{
    
  }
  \begin{minted}[fontsize=\footnotesize]{racket}
    (def-active-token "alias " (str)
      (match-let* (
          [(cons (cons first-alias-end _) _)
            (regexp-match-positions ";" str)]
          [alias-definition (substring str 0 (+ first-alias-end 1))]
          [remaining-lines (substring str (+ 1 first-alias-end))])
        (match alias-definition
          [(regexp
              #px"[\\s]*(.*?)[\\s]*=[\\s]*(.*?)[\\s]*;"
              (list _ alias-name alias-type))
            (regexp-replace*
              (pregexp
                (string-append "(?<![\\w])" alias-name "(?![\\w])"))
              remaining-lines
              alias-type)]
          [str (error (string-append
                "Invalid `alias` syntax in: alias " str))])))

 \end{minted}
\end{frame}

\begin{frame}[fragile]{Java getters and setters}
  \note{
    
  }
  \begin{columns}
    \column{0.4\textwidth}
      \begin{minted}[fontsize=\footnotesize]{java}
        @Setters
        @Getters
        public class Test {
            public String str;
            public int num;
        }
     \end{minted}
    \column{0.65\textwidth}
      \begin{minted}[fontsize=\footnotesize]{java}
        public class Test {
          public String str;
          public int num;

          public void setstr(String str) {
              this.str = str;
          }

          public void setnum(int num) {
              this.num = num;
          }


          public String getstr() {
              return this.str;
          }

          public int getnum() {
              return this.num;
          }
        }
      \end{minted}
  \end{columns}

\end{frame}

\begin{frame}[fragile]{Java getters and setters}
  \note{
    
  }
  
  \begin{enumerate}
    \item Find class limits
    \item Find attributes
    \item Generate getters and setters from the attributes
    \item Insert getters and setters
  \end{enumerate}
\end{frame}

\end{document}
